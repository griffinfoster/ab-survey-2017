% mnras_template.tex
%
% LaTeX template for creating an MNRAS paper
%
% v3.0 released 14 May 2015
% (version numbers match those of mnras.cls)
%
% Copyright (C) Royal Astronomical Society 2015
% Authors:
% Keith T. Smith (Royal Astronomical Society)

% Change log
%
% v3.0 May 2015
%    Renamed to match the new package name
%    Version number matches mnras.cls
%    A few minor tweaks to wording
% v1.0 September 2013
%    Beta testing only - never publicly released
%    First version: a simple (ish) template for creating an MNRAS paper

%%%%%%%%%%%%%%%%%%%%%%%%%%%%%%%%%%%%%%%%%%%%%%%%%%
% Basic setup. Most papers should leave these options alone.
\documentclass[a4paper,fleqn,usenatbib]{mnras}

% MNRAS is set in Times font. If you don't have this installed (most LaTeX
% installations will be fine) or prefer the old Computer Modern fonts, comment
% out the following line
\usepackage{newtxtext,newtxmath}
% Depending on your LaTeX fonts installation, you might get better results with one of these:
%\usepackage{mathptmx}
%\usepackage{txfonts}

% Use vector fonts, so it zooms properly in on-screen viewing software
% Don't change these lines unless you know what you are doing
\usepackage[T1]{fontenc}
\usepackage{ae,aecompl}

%%%%% AUTHORS - PLACE YOUR OWN PACKAGES HERE %%%%%

% Only include extra packages if you really need them. Common packages are:
\usepackage{graphicx}	% Including figure files
\usepackage{amsmath}	% Advanced maths commands
\usepackage{amssymb}	% Extra maths symbols
\usepackage[xindy]{glossaries}
\usepackage{subcaption}
\captionsetup{compatibility=false}

%%%%%%%%%%%%%%%%%%%%%%%%%%%%%%%%%%%%%%%%%%%%%%%%%%

%%%%% AUTHORS - PLACE YOUR OWN COMMANDS HERE %%%%%

% Please keep new commands to a minimum, and use \newcommand not \def to avoid
% overwriting existing commands. Example:
%\newcommand{\pcm}{\,cm$^{-2}$}	% per cm-squared

\newcommand{\GSF}[1]{\noindent\textcolor{blue}{GSF:#1}}

%glossary
\newacronym{alfa}{ALFA}{Arecibo L-Band Feed Array}
\newacronym{dm}{DM}{Dispersion Measure}
\newacronym{frb}{FRB}{Fast Radio Burst}
\newacronym{fwhm}{FWHM}{Full-Width at Half-Maximum}
\newacronym{gbt}{GBT}{Greenbank Telescope}
\newacronym{if}{IF}{Intermediate Frequency}
\newacronym{igm}{IGM}{Intergalactic Medium}
\newacronym{ism}{ISM}{Interstellar Medium}
\newacronym{nip}{NIP}{Non-image Processing}
\newacronym{pll}{PLL}{Phased-locked Loop}
\newacronym{rfi}{RFI}{Radio-frequency Interference}
\newacronym{ska}{SKA}{Square Kilometre Array}
\newacronym{sefd}{SEFD}{System Equivalent Flux Density}
\newacronym{snr}{SNR}{Signal-to-Noise Ratio}
\newacronym{sps}{SPS}{Single Pulse Search}

%%%%%%%%%%%%%%%%%%%%%%%%%%%%%%%%%%%%%%%%%%%%%%%%%%

% Title of the paper, and the short title which is used in the headers.
% Keep the title short and informative.
\title[Verification Tests for FRB Detections]{Verification Tests for FRB
Detections}

\author[ALFABURST Team]{
ALFABURST Team
}
%% The list of authors, and the short list which is used in the headers.
%% If you need two or more lines of authors, add an extra line using \newauthor
%\author[K. T. Smith et al.]{
%Keith T. Smith,$^{1}$\thanks{E-mail: mn@ras.org.uk (KTS)}
%A. N. Other,$^{2}$
%Third Author$^{2,3}$
%and Fourth Author$^{3}$
%\\
%% List of institutions
%$^{1}$Royal Astronomical Society, Burlington House, Piccadilly, London W1J 0BQ, UK\\
%$^{2}$Department, Institution, Street Address, City Postal Code, Country\\
%$^{3}$Another Department, Different Institution, Street Address, City Postal Code, Country
%}

% These dates will be filled out by the publisher
\date{Accepted XXX. Received YYY; in original form ZZZ}

% Enter the current year, for the copyright statements etc.
\pubyear{2017}

% Don't change these lines
\begin{document}
\label{firstpage}
\pagerange{\pageref{firstpage}--\pageref{lastpage}}
\maketitle

% Abstract of the paper
\begin{abstract}
% What is the point of the paper?
% What is the context of the study? What background information is necessary to understand the study?
% How was the study done?
% What is the main take away message?
% What can be said about these results, and how does this affect future work?
The one-off nature of most Fast Radio Bursts (FRBs) requires extra scrutiny in
reporting an astrophysical FRB and triggering automated follow-ups with a
telescope network.  The ALFABURST commensal FRB survey at Arecibo Observatory
has been in operation since June 2015. In that time a number of false-positive
events, which on initial inspect appear to be FRB-like, have been found to be of
local origin. Here we report on one such event as an example of the difficult
challenge of fully automating an FRB search. We discuss observational and
post-processing techniques which are useful to further automate an FRB search
survey.
\end{abstract}

% Select between one and six entries from the list of approved keywords.
% Don't make up new ones.
\begin{keywords}
radio continuum: transients -- methods: observational
\end{keywords}

%%%%%%%%%%%%%%%%% BODY OF PAPER %%%%%%%%%%%%%%%%%%

\section{Introduction}
\label{sec:intro}

In the two years of the initial ALFABURST survey\citep{2017ApJS..228...21C} over
125k 8-second windows have been recorded in which our \gls{frb} search pipeline
detected an event above the minimum \gls{snr} detection threshold. The vast
majority of these events have been due to \gls{rfi}, some of the events are due
to bright single pulses of known pulsars. Sorting through these false positive
events is the main focus of our post-processing methods. Most of the events are
clearly \gls{rfi} which are classified with an automated classifier model. A
small number of windows contain FRB-like events which only on further inspection
of the data, the telescope operation status, and contextual information is it
clear that these events are from local sources.

We expect a number of false-positives to pass our post-processing detection
tests, as we would like to severly limit the potential for type-II errors in our
classifier by accepting a number of type-I errors. So far, all false-positives
we have detected are explainable as relating to the telescope or \gls{rfi} by
examing the observing situation after the fact. This creates a challenge of
automating the triggering of follow-up signalling to other telescopes. Either
there will be an excess of false-positive triggers but with a short delay
between detection and triggering. Or, a non-automated, expert examination of the
event is required to verify, creating a delay in any follow-up.

In this letter we report on FRB-like signals detected with our ALFABURST system,
and discuss how we have improved our post-processing pipleine and observing
system to handle unexpected, but explainable signals. These improvements are
valuable to incorporate into any \gls{frb} search survey.

\section{An example of a FRB-like Signal detected with ALFABURST}
\label{sec:D20161204}

A narrow-in-time, broad-in-frequency, millisecond pulse was detected with the
ALFABURST system at MJD 57726.563263913 / Unix time 1480858266 (09:31:06 Arecibo
local time) in Beam 0 (the central beam) of the \gls{alfa}
receiver\footnote{http://www.naic.edu/alfa/} (Figure
\ref{fig:beam0_dynamic_spec}). ALFABURST was processing 56 MHz of bandwidth
between 1457 MHz and 1513 MHz. The \gls{snr} of this pulse is maximized (10.46)
when the pulse is dedispersed with a \gls{dm} of 293 and the 256 microsecond
resolution is decimated by a factor of 16 to 4 ms time resolution. The
dedispered time series shows an approximately 20 ms \gls{fwhm} pulse. The dip
before and after the pulse is due to the DM-zero removal (i.e. the moving
average is subtracted) during pulse detection. This is a simple way to remove a
drifting gain baseline at the cost of removing some of the overall pulse power,
particually at low DM. The bright, narrow-band signal at approximately $t=0.1$
seconds is locally generated \gls{rfi} which we will cover later in this
section.

\begin{figure*}
    \centering
    % notebooks/event_figures.ipynb
    \begin{subfigure}[t]{0.45\textwidth}
        \centering\captionsetup{width=.95\linewidth}
        \includegraphics[width=1.0\textwidth]{figures/D20161204_buf23_Beam0.pdf}
        \caption{Detected FRB-like event in beam 0 of ALFA. The characteristic dip
        before and after the event is due to zero-DM removal which is part of the
        ALFABURST RFI exciser. The strong, narrowband source at 1480 MHz around 0.1
        s is due to a  local RFI source.  }
        \label{fig:beam0_dynamic_spec}
    \end{subfigure}
    % notebooks/event_figures.ipynb
    \begin{subfigure}[t]{0.45\textwidth}
        \centering\captionsetup{width=.95\linewidth}
        \includegraphics[width=1.0\textwidth]{figures/D20161204_buf4_Beam5.pdf}
        \caption{Detected FRB-like event in beam 5 of ALFA. The event width
        appears wider than the beam 0 event as the zero-DM dips are not as
        prominent.
        }
        \label{fig:beam5_dynamic_spec}
    \end{subfigure}
    \caption{
    Dynamic spectrum (top) and dedispersed time series (bottom) of an FRB-like
    event that was detected simultaneously in beam 0 and 5 of the ALFA receiver
    on December 4, 2016. The dynamic spectrum has been bandpass normalized.
    }
    \label{fig:dynamic_spec}
\end{figure*}

On initial inspection this event looks like a promising new astrophysical
\gls{frb}. The flux density of the event can be computed with the radiometer
equation
%
$$
S = \textrm{SEFD} \frac{\textrm{SNR}}{\sqrt{D \; \Delta \tau \;
\Delta \nu}}
$$
%
using an \gls{sefd} of 3 Jy for the \gls{alfa} receiver. This results in a flux
density of $S = 66$ mJy from Beam 0, which would be lower flux than any
previously detected \gls{frb} \citep{2016PASA...33...45P}. This flux estimate is
an lower limit, as we are assuming the source was at the centre of the beam. The
width is on the high end for \glspl{frb} but still within the range of those
previously reported.

We inspected all other events in the same time window as the Beam 0 event. An
event was found in Beam 5 only (Figure \ref{fig:beam5_dynamic_spec}). This pulse
lines up exactly in time with the Beam 0 event but the \gls{frb} was maximized
(15.99) with a DM=829 dedispersion. Upon further inspection and testing
different DMs for dedispersion we found that this event appeared to narrow in
width at lower DM trials. We see that there was \gls{rfi} clipping in this event
which is known to introduce a bias, resulting in a maximized \gls{snr} at a
different DM trial. The beam 0 and beam 5 event are the same event.

The beam 5 detection has a lower \gls{snr} than the beam 0 detection at DM trial
293. This is still reasonable as the beam 0 sidelobes overlap with all the other
beams, as does the beam 5 sidelobe overlap with the beam 0 primary beam. This
would indicate that the sky source is somewhere between the beam 0 and beam 5
pointing centres. And, the detection was from the edge of the primary lobe or in
the sidelobe of each beam.

The width of the dedispersed pulse in beam 5 appears wider, but this is likely
due to the lower \gls{snr} of the event having a smaller effect on the
spectrum normalization.

One can look at the immediate period before and after the pulse to see that
there are no similar events (Figure \ref{fig:dm_time}). The event appears to be
isolated in time, with a fairly compact representation in DM-space. The event
would be detected with significant \gls{snr} at higher \gls{dm} trials due to
the wide width of the pulse, but peaks at a \gls{dm} trial of 293.

% notebooks/event_figures.ipynb
\begin{figure}
    \includegraphics[width=1.0\linewidth]{figures/D20161204_dmtrials_buf23_Beam0.pdf}
    \caption{DM vs. time plot of for a 1.5 second window centred on the December
    4th event in beam 0. The SNR peaks at a DM of 293. There is a significant
    detection at larger DM trials due to the width of the pulse.
    }
    \label{fig:dm_time}
\end{figure}
%

The Arecibo telescope logging data is reported locally in SCRAM packets which
provide pointing, frequency tuning, and receiver information at approximately
one second resoltuion. From these logs the telescope was pointed at a fixed Dec
(+15:11:28.34) and drifting in RA (event detected at RA=14:42:26.18), i.e. a
fixed (alt,az) pointing during the event. No known pulsar or RRAT is within the
beam at this pointing.

We considered that the observing band could have been changed in that time.  We
have setup the automated system to restart observations when the \gls{if}
frequency is changed.  During the time of the event there was no change in the
\gls{if} during that time.

The SCRAM logs do provide the first indication that this event is due to a local
source. Beyond the pointing and \gls{if}, the SCRAM logs report the position of
the receiver turret and if \gls{alfa} is active. \gls{alfa} is at a position
angle of approximately $26.64^{\circ}$ in the turret, the system reported the
turret was at $206^{\circ}$. \gls{alfa} is not in, or even near the focus.  Our
commensal observation script checks if \gls{alfa} is active before we run
ALFABURST. This is a check on whether the analogue receiver chain is properly
setup for \gls{alfa}, which almost always means that \gls{alfa} is in the focus.
But, as we have found out, there are times when this is not true.

%%%%%%%%%%%%%%%%%%%%%%%%%%%%%%%%%%%%%%%%%%%%%%%%%%%%%
% telescope status
% TODO: dec 4 event: turret in wrong position
The SCRAM logs do not report the active project or observing schedule. During
the time of the event it appears that no receiver was in use, otherwise
\gls{alfa} would not have been active, and we see that \gls{alfa} was
deactivated approximately 20 minutes after the event when a new observation
began.  Looking at the observation schedule for the morning of December 4,
project
P3080\footnote{http://www.naic.edu/vscience/schedule/tpfiles/MichillitagP3080tp.pdf}
was using \gls{alfa} to perform an \gls{frb} survey of the Virgo cluster until
09:00 local time.  After 09:00 local time Project
R3037\footnote{http://www.naic.edu/vscience/schedule/tpfiles/TaylortagR3037tp.pdf}
was scheduled, this is an S-Band RADAR observation.

%\gls{alfa} was not register as switched off until around
%10:00, approximately 15 minutes after the event was detected. This indicated to
%us that the events we are seeing all likely related to the setup of the S-Band
%RADAR.


%%%%%%%%%%%%%%%%%%%%%%%%%%%%%%%%%%%%%%%%%%%%%%%%%%%%%
But, the Beam 5 spectrogram introduces new issues with this event being an
non-terrestrial \gls{frb}. Before and after the event there are events that look
similar to the main event, these event SNRs are maximized by choosing a range of
DMs.

Further, there is a serious red flag when inspecting the bandpass during the
event. The detection band of \gls{alfa} was chosen because it is the most
sensitive region of the band, and relatively flat. But during the event there is
a noticeable shape and slant to the bandpass which is not seen in a typical
observation (Figure \ref{fig:bandpass_response}). The system noise appears
higher, which could indicate an introduction of a stable, warm source to the
system. There is also narrow-in-frequency, periodic RFI at 1468, 1480, 1496,
1504 MHz not usually seen in this band. In the high time and frequency
resolution view these short pulses in 12.5 MHz steps have a characteristic
dampened harmonic oscillation due to frequency locking with a \gls{pll} (Figure
\ref{fig:pll_spectrum}).
%
% notebooks/event_figures.ipynb
\begin{figure}
    \includegraphics[width=1.0\linewidth]{figures/bandpass_response.pdf}
    \caption{Average bandpass response during the Decmber 4, 2016 event for beam
    0 (green) and beam 5 (blue). A typical bandpass (red) is plotted for
    reference. These bandpasses have been normalized in the detection pipeline.
    }
    \label{fig:bandpass_response}
\end{figure}
%

%
% notebooks/event_figures.ipynb
\begin{figure}
    \includegraphics[width=1.0\linewidth]{figures/pll_spectrum.pdf}
    \caption{Dynamic spectrum when a local oscialltor in the receiver dome is
    being locked with a phased-locked loop circuit. This LO is not related to
    the receiver analogue mixing chain, but rather it is associated with RFI
    monitoring equipment.
    }
    \label{fig:pll_spectrum}
\end{figure}
%

Looking at a longer time window before the event we found further indication of
time and frequency-variable structure in the band before the event which looks
related. Approximately 90 seconds before the event we see the band change in
multiple beams, and the appearance of narrow-in-frequency \gls{rfi}.

%But, after inspecting the ancillary data such as the telescope
%status, detections in other beams, observing schedules, and previous in time
%events, we believe this event to have been generated by a local source, mostly
%likely within the receiver dome. 

Currently, we are considering that the event is due to ALFA being covered to
protect against the RADAR system while it was still the active rx. It could be
that this event is due to some standing-wave coupling between the ALFA and the
cover.

% TODO: What is the physical process which generated this event?
% TODO: explanation of terrestial FRB detection
% TODO: follow-up to verify is is a local source, what is the source?

In isolation, and the one-off, transient nature of FRBs make the initial Beam 0
detection look very reasonable. It is only with an extended study of the
meta-data, earlier-in-time evolution of the band, and use of multiple beams to
confirm that this is indeed not an FRB.

\section{Verification Checks}

% TODO:
% low SNR events, due to sudden changes in the noise statistics, expect to
% detect a few of these based in the minimum SNR, what is the time scale for an
% SNR 6 event? these events are due to turrent movement, add flag to filter out
% figure: DM - t plot

% TODO:
% decision tree
% important tests:
%   bandpass check
%   telescope status
%   negative DM statistics
%   previous in time events

% TODO: DM search space: 0-10000, no negative DMs, could reveal RFI by comparing
%  number of events detected in positive and negative DM search space

% TODO: considerations for automated follow-up/VOevents

% TODO: considerations for releasing FRB event data as they are one-off events that
% require extra scrutiny

% TODO:
% future developments in FRB surveys require additional layers of processing.
% large surveys will produce an abundance of false positives. it is cumbersome
% to follow up events now. in the future it will be impossible. additional
% layers should prioritize events by how interesting they are

% TODO: notebook of dec 4 event, low SNR event, make filterbanks available
Jupyter notebooks and information on the filterbanks files are hosted on our
public git repository\footnote{https://github.com/griffinfoster/ab-survey-2017}.

\bibliographystyle{mnras}
\bibliography{../alfaburst.bib} 

% Don't change these lines
\bsp	% typesetting comment
\label{lastpage}
\end{document}

% End of mnras_template.tex
