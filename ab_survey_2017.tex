% mnras_template.tex
%
% LaTeX template for creating an MNRAS paper
%
% v3.0 released 14 May 2015
% (version numbers match those of mnras.cls)
%
% Copyright (C) Royal Astronomical Society 2015
% Authors:
% Keith T. Smith (Royal Astronomical Society)

% Change log
%
% v3.0 May 2015
%    Renamed to match the new package name
%    Version number matches mnras.cls
%    A few minor tweaks to wording
% v1.0 September 2013
%    Beta testing only - never publicly released
%    First version: a simple (ish) template for creating an MNRAS paper

%%%%%%%%%%%%%%%%%%%%%%%%%%%%%%%%%%%%%%%%%%%%%%%%%%
% Basic setup. Most papers should leave these options alone.
\documentclass[a4paper,fleqn,usenatbib]{mnras}

% MNRAS is set in Times font. If you don't have this installed (most LaTeX
% installations will be fine) or prefer the old Computer Modern fonts, comment
% out the following line
\usepackage{newtxtext,newtxmath}
% Depending on your LaTeX fonts installation, you might get better results with one of these:
%\usepackage{mathptmx}
%\usepackage{txfonts}

% Use vector fonts, so it zooms properly in on-screen viewing software
% Don't change these lines unless you know what you are doing
\usepackage[T1]{fontenc}
\usepackage{ae,aecompl}

%%%%% AUTHORS - PLACE YOUR OWN PACKAGES HERE %%%%%

% Only include extra packages if you really need them. Common packages are:
\usepackage{graphicx}	% Including figure files
\usepackage{amsmath}	% Advanced maths commands
\usepackage{amssymb}	% Extra maths symbols
\usepackage[xindy]{glossaries}
\glsdisablehyper
\usepackage{subcaption}
\captionsetup{compatibility=false}

%%%%%%%%%%%%%%%%%%%%%%%%%%%%%%%%%%%%%%%%%%%%%%%%%%

%%%%% AUTHORS - PLACE YOUR OWN COMMANDS HERE %%%%%

% Please keep new commands to a minimum, and use \newcommand not \def to avoid
% overwriting existing commands. Example:
%\newcommand{\pcm}{\,cm$^{-2}$}	% per cm-squared

\newcommand{\GSF}[1]{\noindent\textcolor{blue}{GSF:#1}}

\makeglossaries

%glossary
\newacronym{alfa}{ALFA}{Arecibo L-Band Feed Array}
\newacronym{dm}{DM}{Dispersion Measure}
\newacronym{frb}{FRB}{Fast Radio Burst}
\newacronym{fwhm}{FWHM}{Full-Width at Half-Maximum}
\newacronym{gbt}{GBT}{Greenbank Telescope}
\newacronym{htru}{HTRU}{High Time Resolution Universe}
\newacronym{igm}{IGM}{Intergalactic Medium}
\newacronym{ism}{ISM}{Interstellar Medium}
\newacronym{nip}{NIP}{Non-image Processing}
\newacronym{paf}{PAF}{Phased-Array Feeds}
\newacronym{rfi}{RFI}{Radio-frequency Interference}
\newacronym{ska}{SKA}{Square Kilometre Array}
\newacronym{sefd}{SEFD}{System Equivalent Flux Density}
\newacronym{snr}{SNR}{Signal-to-Noise Ratio}
\newacronym{sps}{SPS}{Single Pulse Search}
\newacronym{vlbi}{VLBI}{Very Long Baseline Interferometry}

%%%%%%%%%%%%%%%%%%%%%%%%%%%%%%%%%%%%%%%%%%%%%%%%%%

\title[The ALFABURST Commensal FRB Survey]{ALFABURST: A commensal search for
Fast Radio Bursts with Arecibo}

\author[G. Foster et al.]{Griffin Foster$^{1,2}$\thanks{E-mail: griffin.foster@physics.ox.ac.uk},
Aris Karastergiou$^{1,3,4}$,
Dan Werthimer$^{2}$,
\and Jayanth Chennamangalam$^{1}$,
Kaustubh Rajwade$^{5}$,
Golnoosh Golpayegani$^{6,7}$,
\and Duncan Lorimer$^{6,7}$,
Mayuresh Surnis$^{6,7}$
\\
% List of institutions
$^{1}$University of Oxford, Sub-Department of Astrophysics, Denys Wilkinson Building, Keble Road, Oxford, OX1 3RH, United Kingdom\\
$^{2}$Department of Astronomy, University of California, Berkeley, 501 Campbell Hall \#3411, Berkeley, CA, 94720, USA\\
$^{3}$Physics Department, University of the Western Cape, Cape Town 7535, South Africa\\
$^{4}$Department of Physics and Electronics, Rhodes University, PO Box 94, Grahamstown 6140, South Africa\\
$^{5}$Jodrell Bank Centre for Astrophysics, University of Manchester, Oxford Road, Manchester M13 9PL, United Kingdom\\ 
$^{6}$Department of Physics and Astronomy, West Virginia University, Morgantown, WV 26505, USA\\
$^{7}$Center for Gravitational Waves and Cosmology, West Virginia University, Chestnut Ridge Research Building, Morgantown, WV 26505, USA\\
}

% These dates will be filled out by the publisher
\date{Accepted XXX. Received YYY; in original form ZZZ}

% Enter the current year, for the copyright statements etc.
\pubyear{2017}
\hypersetup{draft}
% Don't change these lines
\begin{document}
\label{firstpage}
\pagerange{\pageref{firstpage}--\pageref{lastpage}}
\maketitle

% Abstract of the paper
\begin{abstract}
% What is the point of the paper?
% What is the context of the study? What background information is necessary to understand the study?
% How was the study done?
% What is the main take away message?
% What can be said about these results, and how does this affect future work?
The ALFABURST fast radio burst (FRB) survey has been observing commensally with
other projects using the Arecibo L-band Feed Array (ALFA) receiver at the
Arecibo Observatory since July 2015. We report on the non-detection of any FRBs
from that time until August 2017 for a total observing time of 518 hours.
With current FRB rate models, we expected to see multiple **how many??** FRBs based on the
total observing time, telescope sensitivity and beam size. We discuss the
implications of this non-detection of FRBs in the context of recent detections
with other telescopes and the limitation of our search pipeline. **How many pulsars were
detected??** We also report
the discovery of an interesting Galactic radio transient with a DM of 281 pc~cm$^{-3}$, which was detected while the telescope was slewing between fields.
\end{abstract}

% Select between one and six entries from the list of approved keywords.
% Don't make up new ones.
\begin{keywords}
radio continuum: transients -- methods: observational
\end{keywords}

%%%%%%%%%%%%%%%%%%%%%%%%%%%%%%%%%%%%%%%%%%%%%%%%%%

% TODO: detected pulsars: max SNR, SNR-maximized DM, number of pulses

\section{Introduction}
\label{sec:intro}

\glspl{frb} are short-duration, broad-band, dispersed pulses that are detected
at radio frequencies. They are mostly classified by virtue of their dispersion
being far in excess of the expected Galactic contribution. As for radio pulsars,
for FRBs, where we observe the pulse over a frequency band ranging from $\nu_1$ to $\nu_2$,
the resulting dispersion delay 
%
\begin{equation}
\Delta t \propto {\rm DM} \, (\nu_1^{-2} - \nu_2^{-2}),
\end{equation}
%
where the dispersion measure, DM, is the line integral of
the electron column density along the line of sight to the source.

Although the physical process that gives rise to FRBs is unknown, the
possibility that they originate at cosmological distances, and their potential
use as natural probes of large-scale structure and magnetoionic content of the
Universe makes them worthy of attention. They appear as bright sources at
telescopes on Earth, which implies very high luminosities given the distance.
**not sure that they are really that luminous, cf GRBs or GW sources** As short
duration bursts, probably emanating from point-like sources, they offer the
unique opportunities to probe the inter-galactic medium (IGM)
\citep{2013ApJ...776..125M}, as pulsars do for the Galactic interstellar medium.

Since the first reported detection \citep{2007Sci...318..777L}, a
number of surveys using a range of radio telescopes have attempted to
detect further pulses. At the time of writing, 24 \glspl{frb} have
been reported \citep[for an up-to-date list,
  see][]{2016PASA...33...45P}. The majority of these have been
detected with the Parkes Radio Telescope at L-Band
frequencies. FRB121102 was detected in the PALFA survey using Arecibo
Observatory, showing \glspl{frb} are not a phenomenon local to the
Parkes Telescope \citep{2014ApJ...790..101S}. This \gls{frb} is the
only known \gls{frb} to repeat \citep{2016ApJ...833..177S}. FRB110523
was detected with the Green Bank Radio Telescope (GBT) at UHF
frequencies, confirming \glspl{frb} occur at other radio frequencies
outside L-Band \citep{2015Natur.528..523M}.  Recently, a number of
very bright \glspl{frb} has been detected with UTMOST
\citep{2017MNRAS.468.3746C,atel10697} and ASKAP
\citep{2017ApJ...841L..12B}.

Even with the current, small FRB population, it is clear that
properties of \glspl{frb} vary significantly. The measured \glspl{dm}
range from 176 pc cm$^{-3}$ (FRB170827) to 2596 pc cm$^{-3}$
(FRB160102), with pulse widths ranging from sub-ms (unresolved) to
21~ms, and apparent flux densities covering four orders of
magnitude. The sky distribution of these sources appears skewed
towards high galactic latitudes \citep{2015MNRAS.451.3278M}.

So far, \glspl{frb} have been poorly localized as most have been
detected with single dish telescopes. The unknown detection position
in the primary beam, and one-off nature of most of the \glspl{frb}
does also not allow precise determination of the absolute flux density
or the spectral index. Only the repeater FRB121102 has been localized
using \gls{vlbi} \citep{2017ApJ...834L...8M,
  2017ApJ...834L...7T}. Localization is key to understanding
\glspl{frb}. This requires the use of interferometric arrays with
arc-second accuracy, such as CHIME, MeerKAT and ASKAP. 

Apart from localization, \gls{frb} spectra offer important clues on
the nature of the emission process. Low frequency searches with LOFAR
\citep{2015MNRAS.452.1254K}, MWA \citep{2015AJ....150..199T}, and the
GBT \citep{2017arXiv170107457C} have reported non-detections.  No
large surveys for \glspl{frb} have been done above L-band
frequencies. This is, in part, due to the narrowing of beam size which
limits sky coverage.  \cite{2017arXiv170507553L} ran a
coordinated-in-time, multi-telescope campaign of the repeater
\gls{frb}.  They report non-detection of pulses at VHF, C-band,
Ku-band during periods of detected bursts in L-band and
S-band. \cite{atel10675} report detections of FRB121102 from 4-8 GHz
(C-band). In summary, our current understanding of \glspl{frb} spectra
is limited, however they appear not to follow the steep power law
example of radio pulsars, and may even not be smooth and continuous
with frequency.

For single dish telescopes there is a trade-off of sensitivity for
survey speed.  Large beams, while providing a large sky coverage, have
been unsuccessful at detecting \glspl{frb}. Conversely, Arecibo
provides the highest sensitivity, but with a very narrow beam. Parkes
appears to be near the optimal trade-off point between sensitivity and
beam size for this class of object. ASKAP dishes with \glspl{paf}
provide a large sky coverage with a significant enough sensitivity to
detect bright \glspl{frb}. Interferometric arrays such as CHIME and
MeerKAT while provide both sensitivity and sky coverage. One important
question relating to the nature of the \gls{frb} population is what
are the statistics of source numbers versus source flux density, and
whether or not they follow a logN-logS curve that corresponds to
cosmologically distributed standard candles. To answer this question,
it is particularly interesting to sample both extreme ends of the flux
density axis: the brightest \glspl{frb} discovered using small
telescopes in long duration and large sky-coverage surveys, as well as
the weakest \glspl{frb} sampled through high-sensitivity observations
with large telescopes, necessarily sacrificing survey time and sky
coverage. ALFABURST provides the latter.

\section{Survey (July 2015 -- August 2017)}
\label{sec:overview}

ALFABURST is an \gls{frb} search instrument which has been used to commensally
observe since July 2015 with other \gls{alfa} observations at the Arecibo
Observatory. This system is a component of the SETIBURST back-end
\citep{2017ApJS..228...21C} and uses ARTEMIS \citep{2015MNRAS.452.1254K} for
automated, real-time pulse detection. We perform inline RFI removal, baselining
using zero-DM removal \citep{2009MNRAS.395..410E}, and spectrum normalization
before single pulse detection. During this time period a \gls{sps} was
performed from \gls{dm} 0 to 10000, pulse widths from $256~\mu s$ to $16$~ms,
across a 56~MHz bandwidth for all 7 beams. We return to the effective \gls{dm}
of the search in Section \ref{sec:event_rates}. The gain of Arecibo allows for
the most sensitive \gls{frb} search to date.

Detections above a peak \gls{snr} of 10 were recorded along with an
$8.4$~s dynamic spectrum window around the event. When multiple events
were detected in the same time window, these events were pooled
together and recorded to disk.  Approximately 200k 8.4~s datasets were
recorded between July 2015 and August 2017, the vast majority of which
are false detections due to \gls{rfi} signals passing the real-time
\gls{rfi} exciser. We have detected no \glspl{frb} in our commensal
survey.

A wide-feature, learned model was used to classify each dataset in
order to filter out \gls{rfi} and create a priority queue for visual
examination. This model and the post-processing procedures are
discussed in Section \ref{sec:event_classify}. We discuss the expected
event rates in Section \ref{sec:event_rates} and consider possible
explanations for our non-detection result in Section
\ref{sec:discuss}.

%%% SYSTEM VERIFICATION BEGINS %%%

\subsection{Single Pulse Search Verification}
\label{sec:system_verify}

PALFA survey scheduling includes regularly observing known pulsars to
verify timing analysis, this provides a consistent verification of our
\gls{sps} to detect pulses. As the PALFA survey is targeted at the
Galactic plane a number of high \gls{dm} pulsars were observed, single
pulses from B1859+03 (\gls{dm}: 402), B1900+01 (\gls{dm}: 245) (Figure
\ref{fig:B1900}), B2002+32 (\gls{dm}: 234), B1933+16 (\gls{dm}: 158),
among others were detected.

% alfaburst-initial-survey/notebooks/B1900_01.ipynb
\begin{figure}
    \includegraphics[width=1.0\linewidth]{figures/B1900_01.pdf}
    \caption{Detection of a single pulse from Pulsar B1900+01 (DM
      245). The baseline dip before and after the pulse is due to
      zero-DM removal (Reference) }
    \label{fig:B1900}
\end{figure}

%%% SYSTEM VERIFICATION ENDS   %%%

%%% SKY COVERAGE BEGINS %%%

\subsection{Survey Coverage}
\label{sec:survey_coverage}

Since ALFABURST was installed, the majority of ALFA observation time
is allocated for the AGES \citep{2006MNRAS.371.1617A} and PALFA
\citep{2006ApJ...637..446C} surveys (Figure \ref{fig:sky_coverage}).
The AGES survey pointings are off the galactic plane, which is ideal
for \gls{frb} surveys.. PALFA is a pulsar search survey with pointings
near the galactic plane. These lines of sight can introduce
significant dispersion due to the \gls{ism}. We search out to a DM of
10000 which is well beyond the maximum galactic dispersion, but within
the technical capabilities of our system. 

Approximately 65\% of the ALFABURST survey time has been in pointings
out of the galactic plane ($|b| > 5^{\circ}$).  These pointings are
primarily from the ongoing AGES survey.  Pointings in the plane are
primarily from the PALFA survey.  The PALFA survey detected the
repeating \gls{frb} FRB121102 \citep{2014ApJ...790..101S}, the only
\gls{frb} detected with Arecibo thus far.  As ALFABURST has been
running commensally with the PALFA survey since 2015 these two
back-ends act as independent \gls{sps} pipelines, useful for detection
verification.  Since the beginning of ALFABURST observations no
\glspl{frb} have been reported by PALFA. No follow-up observations of
FRB121102 have been conducted using ALFA.

% alfaburst-initial-survey/notebooks/Sky_Coverage.ipynb
\begin{figure*}
    \includegraphics[width=1.0\linewidth]{figures/cartview_sky_coverage.pdf}
    \caption{Sky coverage during ALFA usage between July 2015 and June 2017,
    shown in a Cartesian projection in galactic coordinates. Color represents
    total time pointing in a log scale. The majority of ALFA usage during this
    time was for the PALFA survey along the galactic plane (dot-dashed boxes)
    and the AGES survey (dashed box).  The S-shaped arcs across the plot are due
    to fixed pointings in local azimuth and altitude.
    }
    \label{fig:sky_coverage}
\end{figure*}

%%% SKY COVERAGE ENDS   %%%

%%% OBSERVATION TIME BEGINS %%%

\subsection{Observing Time}
\label{sec:obs_time}

From the beginning of July 2015 to the end of April 2017 \gls{alfa}
has been used for approximately 1400 hours of observing, with all
seven beams functional.  Due to pipeline development and hardware
reliability, ALFABURST was active and functional for, on average, 322
hours per beam.  The current system is setup to be reliably in use for
all beams any time \gls{alfa} is active and in the correct receiver
turret position. Since April 2017 this stable version of the pipeline
has run for an additional 196 hours. This has resulted in a total of
518 hours of processed observing time since ALFABURST began commensal
observations.

%%% OBSERVATION TIME ENDS   %%%

%%% PRIORITIZER BEGINS %%%

\subsection{Event Filter and Prioritizer}
\label{sec:event_classify}

% TODO:
% add figure

The significant DM trial range, variety of \gls{rfi} events, and
commensal nature of the survey, leads to a large number of false
detections. As mentioned above, approximately 200k unique 8.4 second
datasets were recorded with at least one detection above the minimum
peak SNR threshold of 10. For each event window, a diagnostic plot was
generated which contained the original dynamic spectrum, the
dedispersed dynamic spectrum of the SNR-maximized DM, along with a
frequency collapsed time series of the detection. A set of statistics
for each of these plots was also computed when this figure was
generated.

A sample of the datasets were labelled into 8 categories of RFI,
systematic effects, and astrophysical source (pulsars). Classification
using these labels has not been constant throughout the survey due to
software improvement, and changes in telescope observing strategy.

Pulses from known pulsars were used as a proxy category for the FRB
class. The number of astrophysical pulse detections was low compared
to the number of false-positive detections. It was necessary to use a
large number of categories as RFI and systematic effects took on a
variety of forms.  This had the additional effect of balancing out the
number of events in each class, making model training more robust.

These statistics along with the labels were used to build a random
forest probabilistic classifier model. All unlabelled datasets were
given a probability of belonging the predefined categories. In
training the model we optimized for a high recall, since in searching
for \glspl{frb} we are inclined to allow for a large number of
false-positive events (detection due to RFI or systematics) as long as
there are no false-negative events (Pulses classified as RFI).

Using a probabilistic multi-label classifier allows us to prioritize
the order and amount of time we spend on examining event
datasets. Those with high probability of belonging to a single class
can be examined as a group quickly.  Datasets which fall into multiple
classes are examined more thoroughly, they are labelled by hand, and
the set of features extracted during the figure generation process is
refined to further differentiate classes. This model building,
prioritizing, and examination process was iterated on multiple times
to improve the classifier. We continue to iterate on this model and
will use it for future prioritization of examining events.

We have used our classifier model as a data exploration tool to add and refine
procedural filters to the data. We have not used the classifier model directly
in our pipeline as the black-box nature of the model can lead to
misclassification.  During the offline data examination process a number of
simple filters were developed to cut down on the number of false-positive
detections without relying on the classifier model. For example, datasets in
which the maximum DM of events was less than 50 were cut. And, given the
SNR-maximized DM trial $DM_{\textrm{SNR}}$, if the DM range exceeds $(0.5 \times
DM_{\textrm{SNR}}, 1.5 \times DM_{\textrm{SNR}})$, then the event is due to long
duration RFI and is cut.  Applying the various filters we reduced the number of
datasets to approximately 30k. The windows were sorted by SNR, and the top SNR
events were examined first.  During this process all datasets were labelled.
Astrophysical events were identified based on the beam ID and pointing
information.

%%% JUNE 18, 2017 BEGINS %%%

\section{The event of 2017, June 18}
\label{sec:18062017}

Though we report no detection of FRBs in the first two years of
observations with ALFABURST we have made an initial detection of an as
yet unknown broad-band (within our band) pulse (Figure
\ref{fig:D20170618_spectrum}) at a peak SNR of 18. The peak SNR is
maximized by dedispersion using a DM of 281~pc~cm$^{-3}$ and time
decimation factor 8. The pulse width is approximately 3 ms wide. The
pulse occurred in beam 5, and there were no other detections in the
other beams at the time.

% watermark:/home/griffin/data/alfa/Beam5_fb_D20170618T005616.buffer2-paper.ipynb
\begin{figure}
    \includegraphics[width=1.0\linewidth]{figures/Beam5_fb_D20170618T005616_buffer2_spectrum.pdf}
    \caption{A broad band pulse (SNR maximized at DM=281) detected in beam 5
    while the telescope was slewing during a PALFA observation. There is no
    known source which has been associated with this detection. As the
    observation was in the galactic plane it is likely galactic in origin.
    }
    \label{fig:D20170618_spectrum}
\end{figure}

% TODO: dispersion fitting

The shape of the pulse is made up of two clear components, with the
secondary pulse arriving approximately 20 ms after the primary pulse,
as seen in the dynamic spectrum (Figure
\ref{fig:D20170618_spectrum}). In DM-time space the event is compact,
consistent with a $\nu^{-2}$ dispersion relation (Figure
\ref{fig:D20170618_dmspace}).

% watermark:/home/griffin/data/alfa/Beam5_fb_D20170618T005616.buffer2-paper.ipynb
\begin{figure}
    \includegraphics[width=1.0\linewidth]{figures/Beam5_fb_D20170618T005616_buffer2_dmspace.pdf}
    \caption{DM-time plot of the June 18, 2017 pulse. The pulse is compact in
    DM-time space, consistent with an astrophysical event. The secondary pulse
    20 ms after the primary pulse causes the intensity to be slightly elongated
    to higher trial DMs.
    }
    \label{fig:D20170618_dmspace}
\end{figure}

The detection occurred at 00:56:16 AST on 2017, June 18 (MJD
57922.2100381) during a PALFA observation. This detection was not
reported by the PALFA collaboration as it occurred when the telescope
was slewing between fields. Data during this time are not recorded
with PALFA. This is the first known detection of a transient,
broad-band pulse using ALFA during such a slew. However. this makes it
challenging to determine the accurate source position. Pointing
information from Arecibo is reported every second.  During the
detection the pointing was changing by approximately 5 arcminutes per
second in RA 2 arcminutes per second in DEC. This rate gives us a
conservative estimate of the error in pointing at the time the pulse
was detected. Based on the time stamp of the pulse and the pointing
data the pulse occurred when beam 5 of ALFA was pointing at RA: 18~h
45~$\pm 5$~m, Dec: +00 d 38'~$\pm 2$ (galactic coordinates $l: 32.7812, ~b:
+1.6850$).

% TODO: pointing error refinement

This beam 5 pointing is close to the Galactic plane in the first quadrant. The
DM distance estimated from the NE2001 model \citep{2002astro.ph..7156C} is
approximately 6 kpc, which is well within the Galaxy. A search of the ATNF
pulsar database\footnote{http://www.atnf.csiro.au/people/pulsar/psrcat/}
\citep{2005AJ....129.1993M}, RRAT
catalog\footnote{http://astro.phys.wvu.edu/rratalog/}, and recent PALFA
discoveries \footnote{http://www.naic.edu/~palfa/newpulsars/} revealed no known
source with a DM near 281 pc cm$^{-3}$ within a degree of the pointing.

As the telescope was slewing at the time, the source was only in the
primary lobe for a fraction of a second (assuming it was in the
primary lobe and not a side lobe). It could therefore be an RRAT which
we serendipitously detected at the correct moment, or it could be an
individual pulse from a pulsar. This region has been previously
surveyed with PALFA and the Parkes Multi-beam Survey
\citep{2001MNRAS.328...17M} with no significant detection of a pulsar
at this DM.

The pulse appears brighter at higher frequencies, which could be due
to scintillation. Another reason for this frequency-dependent
structure is that the pointing of the telescope is changing during the
total dispersion time of the pulse within the observed band. As the
pointing moves, the corresponding telescope gain also changes.  There
was a higher beam gain at the beginning of the pulse compared to the
end of the pulse, inducing spectral colorization.

Follow-up observation of this region is planned. We are assuming the source was
within the primary lobe of beam 5 of ALFA. The observation plan is to cover a
larger bandwidth, tile the beam to account for the pointing ambiguity, and
perform a single pulse and harmonic search.

%%% JUNE 18, 2017 ENDS %%%

%%% EVENT RATES BEGINS %%%

\section{Expected FRB Events}
\label{sec:event_rates}

The currently known 24 FRBs vary significantly in \gls{dm}, pulse
width, and flux density. Despite this, we assume a simple model to
derive an expected event rate with our survey.  We use a model
\citep{2013MNRAS.436L...5L} which assumes \gls{frb} sources are
standard candles with a fixed spectral index, uniformly distributed in
co-moving volume. The event rates in this model are scaled to the
event rates reported in \cite{2013Sci...341...53T}. 

Taking advantage of the large forward gain of Arecibo, we account for
the sensitivity of the 7 ALFA beams out to outer edge of the first
sidelobe. In practice we do this by splitting the beam and first
sidelobe into shells of progressively lower gain but larger sky
coverage, and integrate to obtain the totals.

An \gls{alfa} beam is approximately 3.8' x 3.3' at \gls{fwhm} across
the band.  The ALFA beam size is known to be relatively fixed in size
across the band due to the optics \citep{GALFAbeam}.  Given the
average observing time per beam of 518 hours this results in a survey
coverage of $\sim 10 \; \textrm{deg}^2 \; \textrm{hours}$ when
accounting for all 7 beams. This is a small survey coverage compared
to most other \gls{frb} surveys, primarily due to the narrow beam size
of Arecibo. The combined Parkes multi-beam surveys have a total of
8231 observation hours \citep{2016MNRAS.460.3370C}, and a FWHM survey
metric of $\sim 4500 \; \textrm{deg}^2$ hours.  ALFABURST does not
compete with other surveys on sky coverage, rather it competes on
sensitivity.  Using Equation 6 of \cite{2015MNRAS.452.1254K}, an
\gls{sps} pipeline is sensitive to pulses with a minimum flux density
(in Jy) of
%
\begin{equation}
S_{min} = \textrm{SEFD} \frac{\textrm{SNR}_{min}}{\sqrt{D \; \Delta \tau \;
\Delta \nu}}
\end{equation}
%
which is a function of the telescope \gls{sefd}, the minimum \gls{snr}
detection level $\textrm{SNR}_{min}$ and the decimation rate $D$
compared to the native instrumental time resolution $\tau$, this comes
from the search pipeline which averages together spectra to search for
scattered pulses. ALFABURST has a native resolution of $\Delta \tau =
256 \; \mu s$, effective bandwidth $\Delta \nu = 56 \textrm{MHz}$, and
$\textrm{SNR}_{min} = 10$. The FWHM \gls{sefd} of the \gls{alfa}
receiver is approximately 3 Jy across the band for all beams.

The \gls{sps} pipeline is configured to search for pulses from 256~$\mu$s to 16
ms. Considering only the main beam lobe, a perfect matched filter would result
in a sensitivity to pulses with a minimum flux of $S_{256 \mu\textrm{s}} = 250$
mJy to $S_{16 \; \textrm{ms}} = 31$ mJy \citep{2015MNRAS.452.1254K}. Figure
\ref{fig:fwhm_sefd_z} shows the peak flux density of using the standard candle
\gls{frb} model as a function of source redshift for different model spectral
indices. The dashed lines of constant flux show the sensitivity of the ALFABURST
search pipeline to pulses of different widths. Assuming a positive spectral
index model ($\alpha=1.4$) results in a sensitivity out to the maximum
redshift/\gls{dm} for pulses with widths of at least 1 ms. A flat spectral index
model results in sensitivity from $z \sim 1.5$ (256~$\mu$s) out to $z \sim 5$
(16~ms) depending on pulse width. A negative spectral index model ($\alpha \sim
-1.4$) limits the survey to $z < 3$ for all pulse widths.

% alfaburst-initial-survey/notebooks/ALFABURST_Derived_FRB_Rates.ipynb
\begin{figure}
    \includegraphics[width=1.0\linewidth]{figures/fwhm_sefd_z_relation.pdf}
    \caption{Sensitivity of the ALFABURST search pipeline (dashed) to FRB pulses
    assuming a standard candle model using different spectral index models
    (solid).
    }
    \label{fig:fwhm_sefd_z}
\end{figure}

If we assume a simple model of $\alpha=0$ as we have limited
information about the source spectral index, and a pulse width of 4 ms
as that is an approximate median pulse width of reported \glspl{frb},
then this results in a maximum redshift of $z=3.4$ (a co-moving
distance of 6.8 Gpc) and a survey volume of $6 \times 10^5$ Mpc$^3$
when using all 7 \gls{alfa} beams. The number of galaxies sampled in
this volume is $6 \times 10^3$ assuming a constant galaxy number
density of $10^{-2}$ per Mpc$^3$.  The volumetric event rate from
\cite{2013Sci...341...53T} is stated to be $R_{\textrm{FRB}} =
10^{-3}$ \glspl{frb} per galaxy per year. With these assumptions, we
should expect to detect $\sim 1$ \glspl{frb} based on the current
observation time. We note once again that this calculation is only
based on the sensitivity and size of the main beam lobe.

As mentioned above, it is worth also taking into account the entire
first side lobes of the beams as Arecibo would be sensitive to detect
most previous \glspl{frb} in these. Using the parameterized \gls{alfa}
beam model (Figure \ref{fig:alfa_beam}) \citep{GALFAbeam} we can
compute the \gls{frb} survey metric and expected rates as a function
of beam sensitivity.  The first side lobes peak at around $-10$ dB and
provide a significant increase in sky coverage compared to just the
primary lobes.

% alfaburst-initial-survey/notebooks/ALFABURST_Derived_FRB_Rates.ipynb
\begin{figure}
    \includegraphics[width=1.0\linewidth]{figures/ALFA_beam_1425MHz_dB.pdf}
    \caption{Primary and first side lobe model of the AFLA receiver in
    decibels, cut-off at $-30$ dB.The first side lobe peak at around $-9$ dB.
    }
    \label{fig:alfa_beam}
\end{figure}

The total survey metric can be computed as a function of the beam
sensitivity by integrating over the beam (Figure
\ref{fig:survey_metric_sense}). We convert the beam model to units of
Jy by assuming that the $-3$ dB point corresponds to the \gls{fwhm}
SEFD of 3~Jy. The survey metric increases to approximately $30 \;
\textrm{deg}^2$ hours by including more of the primary beam beyond the
\gls{fwhm} point. The steep further increase in the survey metric seen
in Figure \ref{fig:survey_metric_sense} arises from including the
first side lobes. The long tail comes from the residual sensitivity by
integrating over the remaining beam.

% alfaburst-initial-survey/notebooks/ALFABURST_Derived_FRB_Rates.ipynb
\begin{figure}
    \includegraphics[width=1.0\linewidth]{figures/full_survey_metric_sense.pdf}
    \caption{Survey metric as a function of the ALFA receiver minimum
    sensitivity using the ALFA primary and first side lobes. The $-9$ dB point
    (green circle) which is the beginning of the first side lobe sensitivity and
    $-12$ dB point (red square) which is the FWHM of the first side lobe are marked.
    }
    \label{fig:survey_metric_sense}
\end{figure}

The survey volume is significantly increased by including a large
portion of the beam. It is not possible to put together a figure
similar to Figure \ref{fig:fwhm_sefd_z} when considering the full
beam. It is however possible, under the assumption of flat intrinsic
FRB spectra, to compute the maximum redshift as a function of beam
size and sensitivity. Plotting the survey metric as a function of
maximum redshift (Figure \ref{fig:full_sefd_z}) shows how the full
beam model increases the survey metric as a function of redshift. The
total survey volume is computed by integrating over redshift.
Including additional ALFA side lobes results in minimal increase in
the survey volume.

% alfaburst-initial-survey/notebooks/ALFABURST_Derived_FRB_Rates.ipynb
\begin{figure}
    \includegraphics[width=1.0\linewidth]{figures/full_sefd_z_relation.pdf}
    \caption{Survey metric as a function of redshift using the standard candle
    model with a flat spectral index ($\alpha=0$) and pulse width of 4 ms. The
    bump out to $z=1.5$ is due to the including the ALFA first side lobes.
    Markers indicate the $-9$ dB (green circle) and $-12$ dB (red square) of the
    ALFA beam.
    }
    \label{fig:full_sefd_z}
\end{figure}

The integrated survey volume out to the first side lobe is $5.8 \times
10^6$ Mpc$^3$. The expected number of \glspl{frb} in the survey is
$\sim 3$ when using the galaxy number density and $R_{\textrm{FRB}}$
stated above. Though this event rate is more complex to model, it
provides a more accurate assessment of the expected detection rates
based on the apparent flux of previously reported \glspl{frb}. We note
that the volume we are sampling is biased towards small distances, due
to the large sky coverage of the low sensitivity outer parts of the
beam.

%%%%%%%%%%%%%%% stopped here

We should note, that 




In the analysis above, we have not provided error estimates to our
total expected events. We are assuming standard candle FRBs and flat
spectra, assumptions which are impossible to attribute error margins
to. The significance of our result is difficult to quantify, beyond
stating that, under reasonable assumptions and previous detections, we
would have expected an FRB in our survey. However, the non-detection
cannot be assumed to imply significant discrepancy with our current
understanding of this population.

% TODO: split apart
\subsection{Sensitivity Upper Limit}
\label{sec:upper_limit}

The sensitivity limit of ALFABURST is determined by the telescope
sensitivity and receiver beam model. However, there is also a
sensitivity upper limit due to the real-time parameterized \gls{rfi}
exciser. The choice of these parameters sets an upper limit on the
flux of a pulse before it a portion of the flux is clipped and
replaced. Individual frequency channels in a spectra are replaced when
they exceed a threshold $T_{\textrm{chan}}$ after the spectra is
normalized ($\mu=0$, $\sigma=1$). And, entire spectra are clipped when
the summed spectra exceeds a threshold $T_{\textrm{spectra}}$. For
standard ALFABURST operation $T_{\textrm{chan}} = 5$ and
$T_{\textrm{spectra}} = 10$.

For very bright, small \gls{dm} pulses the \gls{rfi} exciser will replace
channels or spectra, reducing the overall flux or potentially removing the
entire pulse.  For bright, high \gls{dm} pulses the spectra will likely not be
replaced, but individual channels may be, resulting in a lower detected flux.
The \gls{rfi} exciser only works in the undecimated-in-time case ($D=1$), that
is the sensitivity that we are most concerned about when setting the \gls{rfi}
exciser thresholds.

\glspl{frb} detected at L-Band range between 200 mJy to a few Jy in flux
density, typically on the order of 5-10 milliseconds. For the sensitivity of the
\gls{alfa} receiver, individual channels of flux greater than 2.8 Jy will be
flagged, this will not have an effect on our ability to detect even the
brightest \glspl{frb}.  The maximum integrated pulse flux ($256 \; \mu$s width,
\gls{dm} $=0$) is $\sim250$ mJy before the pulse is clipped. The maximum
detectable flux increases as the square-root of the pulse width.  We see in
verification observations of bright, low \gls{dm} pulsars that individual
pulses are often excised. But, as we are interested in detecting high \gls{dm}
\glspl{frb} we have a higher upper limit as the flux is spread over multiple
spectra.

For reference, the minimum \gls{dm} of a pulse before the at least one channel
is shifted to the next spectra in time is \gls{dm} $=1.8$ pc cm$^{-3}$ for the
typical ALFABURST observing band (using Eq. 5.1 of \cite{2004hpa..book.....L}).
And, the minimum \gls{dm} before each frequency channel is in a separate spectra
is \gls{dm} $=976$ pc cm$^{-3}$. Most reported \glspl{frb} fall with in this
\gls{dm} range, so we consider a test \gls{frb} with a dispersion measure of 250
pc cm$^{-3}$ and narrow pulse width of $256 \; \mu$s to report our survey
upper-limit sensitivity. A \gls{dm} of 250 pc cm$^{-3}$ results in approximately
$1/128$ of the pulse per spectra. A bright pulse ($>32$ Jy) would be excised as
\gls{rfi} in this test case. This an extreme case, as most \glspl{frb} are wider
in width and at higher \glspl{dm}.  Our pipeline would preserve the flux of all
detected \glspl{frb} except the extreme FRB150807, and possibly FRB170827
(Figure \ref{fig:sensitivity_range}).  This also assumes the \gls{frb} is
detected at boresight, we would still be sensitive to such bright pulses in the
side lobes.

Figure \ref{fig:sensitivity_range} shows that ALFABURST sensitivity region based
on pulse width and peak flux, assuming detection at boresight. The ALFABURST
sensitivity region (purple) indicates the survey would be able to detect the
vast majority of previously reported \glspl{frb}. The upper flux limit cut-off
due to \gls{rfi} mitigation would have clipped out some flux from the extremely
bright, narrow, and low-DM FRB150807. Though, a portion of the flux would have
remained, and the \gls{frb} would have still been detected as significant.
Recent detections with UTMOST \citep{2017MNRAS.468.3746C,atel10697} indicate
that the parameter space in pulse width should be extended.  FRB160317 has a
measured pulse width of 21 ms. Currently the pipeline decimates in time out to
16 ms. The pipeline still sensitive to wider pulses, but at a loss in \gls{snr}
as indicated in the slight slope on the right side of the purple region of
Figure \ref{fig:sensitivity_range}, but we would not have detected FRB160317. It
is a minor change to our pipeline to increase the maximum pulse-width search.

% alfaburst-initial-survey/notebooks/Fluence_Rate.ipynb
\begin{figure}
    \includegraphics[width=1.0\linewidth]{figures/sensitivity_range.pdf}
    \caption{ALFABURST single pulse sensitivity (purple region). Automated RFI
    excision excludes narrow in width, low DM, bright FRBs such as FRB150807
    (yellow region).  Previously detected FRBs from Parkes (black triangle), GBT
    (red circle), Arecibo (white diamond), UTMOST (teal pentagon), and ASKAP
    (yellow-green hexagon) are plotted for reference. Line of constant fluence
    (solid) are plotted for reference. The fluence completeness (dashed) is 0.5
    Jy ms out to pulse widths of 16 milliseconds.
    }
    \label{fig:sensitivity_range}
\end{figure}

%%% EVENT RATES ENDS   %%%

%%% FLUENCE RATE BEGINS %%%

%\subsection{Fluence Completeness}
%\label{sec:fluence}

The fluence completeness of the survey \citep{2015MNRAS.447.2852K} is determined
by the minimum detectable fluence at the maximum sampled pulse width in the
survey. ALFABURST has a fluence completeness of $0.5$ Jy ms up to a pulse width
of 16 ms (Figure \ref{fig:sensitivity_range}). All previously reported FRBs are
within this completeness sample except for those noted in Section
\ref{sec:upper_limit}.

%%% FLUENCE RATE ENDS   %%%

\section{Discussion}
\label{sec:discuss}

Multiple factors could be contributing to our non-detection with the ALFABURST
survey. We derived an expected event rate based on the telescope sensitivity,
observing time, and a standard candle model \citep{2013MNRAS.436L...5L}. This is
a simple model based on the empirical event rates from detections in the
\gls{htru} survey \citep{2013Sci...341...53T}. The repeating nature of
FRB121102 indicates that there could be multiple classes of \gls{frb}
progenitors, or this standard candle model does not accurately model event
rates. The detection of bright, high-DM \glspl{frb} with ASKAP
\citep{2017ApJ...841L..12B} and UTMOST \citep{2017MNRAS.468.3746C,atel10697}
might indicate that \glspl{frb} are not standard candles.

The limited processing bandwidth of ALFABURST may be a cause of the survey
non-detection. Multiple detected \glspl{frb} show apparent scintillation and
steep spectral indices. It is not possible to differentiate between an apparent
spectral index induced by the beam or an absolute spectral index from the
source. Though, the localization and repeated detections of FRB121102 show
there is significant spectral variation from the source or due to the
intervening medium. Other \glspl{frb} show frequency-dependent structure which
could be due to beam colorization, intrinsic structure, or due an intermediate
effect. Plasma lenses in the \gls{frb} progenitor host galaxy could be
modulating the pulse amplitude as a function of frequency and time (if the
source repeats) \citep{2017ApJ...842...35C}. This effect introduces an additional 
uncertainty in the \gls{frb} rate modeling as the apparent spectral indices of detected
\glspl{frb} may not be intrinsic. Thus, the observed frequency structure in an
\gls{frb} (repeating or not) would be dependent on multiple factors including observing
frequency, bandwidth, epoch, and even sky direction. If an \gls{frb} did occur in the
field of view of the telescope while ALFABURST was in operation we could have
been unlucky and scintillation or lensing caused the pulse in the band to go
below the detection threshold. Assuming no scintillation or lensing, an
increase to the full \gls{alfa} band would result in a $\sqrt{6}$ increase in
sensitivity. But, also important is a more complete sampling of the frequency
space if these effects are modulating the pulse.

\cite{2015MNRAS.451.3278M} conclude that the apparent deficit of \glspl{frb} at
low galactic latitudes is due to diffractive interstellar scintillation. Their
model shows that the true event rate is a factor of $\sim 4$ lower than the rate
reported in \cite{2013Sci...341...53T}, which the rate used in the standard
candle model \citep{2013MNRAS.436L...5L}. Though the ALFABURST survey is evenly
split across high and low galactic latitudes.  \cite{2015MNRAS.451.3278M} predict
that the increase in sensitivity of using Arecibo compared to Parkes should
result in a factor of 14 increase in detections, assuming a similar bandwidth
($\sim 300$ MHz). Accounting for the smaller bandwidth of ALFABURST means there
should still be a factor of a few increase in rates. This non-detection result
indicates that the \cite{2015MNRAS.451.3278M} flux density distribution is not
as steep as predicted.

The sensitivity of Arecibo allows the ALFABURST survey to probe a search volume
out to higher red-shifts than other surveys. In the standard candle model a flat
luminosity function is assumed. If there is a peak similar to the star formation
rate around $z=2$ \citep{2014ARA&A..52..415M} than the expected event rate
should be lower. Similarly, if \glspl{frb} are intrinsically steep spectra then
distant \glspl{frb} will have a decreased observed flux at L-band, possibly
below the search sensitivity threshold. \cite{2017arXiv170507553L} report
FRB121102 to be band limited during simultaneous observation campaigns using
multiple telescopes to cover a broad range of the radio band. \cite{atel10675}
observed 15 pulses from FRB121102 across the 4-8 GHz band and reported spectral
variation over a brief period of time. A high redshift, band-limited \gls{frb},
which ALFABURST is sensitive to, could be shifted below L-band. Such a pulse
would not be detected with ALFABURST.

% Possible reasons for non-detection:
%   * standard candle model:
%   * scintillation:
%       * limited bandwidth, repeater band varies
%       * relation to Macquart and Johnston paper: rate off/on the plane is
%       different due to scintillation regimes
%       * Macquart and Johnston: discount scintillation effects, they make predictions that
%       a number of FRBs should be detected with Arecibo. Our non-detection refutes this.
%       * repeater and low freq frb searches: spectra are not pulsar like. may be
%       due to strong scintillation
%   * plasma lens model (cordes et al.):
%       * narrow bandwidth
%   * Star formation rate:
%       * implied FRB Luminosity function: cosmological star formation rate, peaks
%       around z=2, Madau & Dickinson 2014
%   * steep spectrum/band limited:
%       * Law et al see no detections at low and 5 GHz+ freqs -> band limited,
%       steep spectrum ; we are probing a large z, perhaps the band is shifted
%       out of L-band
%       * FRB121102 shows dramatic frequency structure over wide bands

\section{Future Work}
\label{sec:future_work}

The ALFABURST system will continue to run commensally with other ALFA projects,
leading to an improvement on the event rate of low-fluence \glspl{frb}.  The
current \gls{sps} pipeline is undergoing a significant upgrade. The input
bandwidth is limited to 56 MHz of the full 336 MHz digital band due to IO
limitations. A new pipeline developed for \gls{ska} \gls{nip} will be used to
process the full \gls{alfa} band.  This will increase sensitivity, and improve
detection rates for scintillating or lensed \glspl{frb}.  An improved version of
the real-time \gls{rfi} exciser is currently being developed and will be
deployed to reduce the false detection rate. The post-processing classifier and
prioritizer model is being updated to make use of an auto-encoder to select deep
features and auto-generate classes. This will allow for an improved follow-up
and analysis cycle.

Over the time period ALFABURST has been active, the use of \gls{alfa} has
decreased as the PALFA and AGES surveys end. The 327 MHz and L-band wide feeds
are commonly used. We are generalizing the \gls{alfa} specific \gls{sps}
pipeline to be used when these feeds are active, increasing our survey time and
sampling a larger portion of frequency space. Additionally, our search pipeline
will be duplicated for use on the \gls{gbt} to be commensally run with L-band
observations. 

Jupyter notebooks are hosted on our public git
repository\footnote{https://github.com/griffinfoster/alfaburst-initial-survey}.

\bibliographystyle{mnras}
\bibliography{alfaburst.bib} 

\bsp	% typesetting comment
\label{lastpage}
\end{document}

% End of mnras_template.tex
