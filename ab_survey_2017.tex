% mnras_template.tex
%
% LaTeX template for creating an MNRAS paper
%
% v3.0 released 14 May 2015
% (version numbers match those of mnras.cls)
%
% Copyright (C) Royal Astronomical Society 2015
% Authors:
% Keith T. Smith (Royal Astronomical Society)

% Change log
%
% v3.0 May 2015
%    Renamed to match the new package name
%    Version number matches mnras.cls
%    A few minor tweaks to wording
% v1.0 September 2013
%    Beta testing only - never publicly released
%    First version: a simple (ish) template for creating an MNRAS paper

%%%%%%%%%%%%%%%%%%%%%%%%%%%%%%%%%%%%%%%%%%%%%%%%%%
% Basic setup. Most papers should leave these options alone.
\documentclass[a4paper,fleqn,usenatbib]{mnras}

% MNRAS is set in Times font. If you don't have this installed (most LaTeX
% installations will be fine) or prefer the old Computer Modern fonts, comment
% out the following line
\usepackage{newtxtext,newtxmath}
% Depending on your LaTeX fonts installation, you might get better results with one of these:
%\usepackage{mathptmx}
%\usepackage{txfonts}

% Use vector fonts, so it zooms properly in on-screen viewing software
% Don't change these lines unless you know what you are doing
\usepackage[T1]{fontenc}
\usepackage{ae,aecompl}


%%%%% AUTHORS - PLACE YOUR OWN PACKAGES HERE %%%%%

% Only include extra packages if you really need them. Common packages are:
\usepackage{graphicx}	% Including figure files
\usepackage{amsmath}	% Advanced maths commands
\usepackage{amssymb}	% Extra maths symbols
\usepackage[xindy]{glossaries}

%%%%%%%%%%%%%%%%%%%%%%%%%%%%%%%%%%%%%%%%%%%%%%%%%%

%%%%% AUTHORS - PLACE YOUR OWN COMMANDS HERE %%%%%

% Please keep new commands to a minimum, and use \newcommand not \def to avoid
% overwriting existing commands. Example:
%\newcommand{\pcm}{\,cm$^{-2}$}	% per cm-squared

\newcommand{\GSF}[1]{\noindent\textcolor{blue}{GSF:#1}}

\makeglossaries

%glossary
\newacronym{alfa}{ALFA}{Arecibo L-Band Feed Array}
\newacronym{frb}{FRB}{Fast Radio Burst}
\newacronym{igm}{IGM}{Intergalactic Medium}
\newacronym{ism}{ISM}{Interstellar Medium}
\newacronym{sefd}{SEFD}{System Equivalent Flux Density}
\newacronym{snr}{SNR}{Signal-to-Noise Ratio}
\newacronym{sps}{SPS}{Single Pulse Search}

%%%%%%%%%%%%%%%%%%%%%%%%%%%%%%%%%%%%%%%%%%%%%%%%%%

\title[A Low-Flux FRB Rate Limit using ALFA]{A Low-Flux FRB Rate Limit using ALFA}

\author[ALFABURST Team]{
ALFABURST Team
}

% These dates will be filled out by the publisher
\date{Accepted XXX. Received YYY; in original form ZZZ}

% Enter the current year, for the copyright statements etc.
\pubyear{2017}

% Don't change these lines
\begin{document}
\label{firstpage}
\pagerange{\pageref{firstpage}--\pageref{lastpage}}
\maketitle

% Abstract of the paper
\begin{abstract}
% What is the point of the paper?
% What is the context of the study? What background information is necessary to understand the study?
% How was the study done?
% What is the main take away message?
% What can be said about these results, and how does this affect future work?
The ALFABURST fast radio burst (FRB) survey has been observing commensally with
other projects using the ALFA receiver since July 2015. We report on the
non-detection of any FRBs from that time until May 2017. With current FRB rate
models we expected to see multiple FRBs in based on the total observing time,
telescope sensitivity and beam size. We discuss the implications for this
non-detection FRBs in the context of recent detections with other telescopes.
\end{abstract}

% Select between one and six entries from the list of approved keywords.
% Don't make up new ones.
\begin{keywords}
radio continuum: transients -- methods: observational
\end{keywords}

%%%%%%%%%%%%%%%%%%%%%%%%%%%%%%%%%%%%%%%%%%%%%%%%%%

\section{Introduction}

ALFABURST is an \gls*{frb} search instrument which has been used to commensally
observe since July 2015 with other \gls*{alfa} observations at Arecibo
Observatory. This system is a component of the SETIBURST backend
\citep{2017ApJS..228...21C} and uses ARTEMIS \citep{2015MNRAS.452.1254K} for
automated, real-time detection.

%%% OBSERVATION TIME BEGINS %%%

%TODO: observation time

%%% OBSERVATION TIME ENDS   %%%

%%% SENSITIVITY BEGINS %%%

Using Equation 6 of \cite{2015MNRAS.452.1254K}, a \gls*{sps} pipeline is
sensitive to pulses with a minimum flux density (in Jy) of
%
\begin{equation}
S_{min} = \textrm{SEFD} \frac{\textrm{SNR}_{min}}{\sqrt{D \; \Delta \tau \;
\Delta \nu}}
\end{equation}
%
which is a function of the telescope \gls*{sefd}, the minimum \gls*{snr}
detection level $\textrm{SNR}_{min}$ and the decimation rate $D$ compared to the
native instrumental time resolution $\tau$, this comes from the search pipeline
which averages together spectra to search for scattered pulses. ALFABURST has a
native resolution of $\Delta \tau = 256 \; \mu s$, effective bandwidth $\Delta
\nu = 56 \textrm{MHz}$, and $\textrm{SNR}_{min} = 10$. The \gls*{sefd} of the
\gls*{alfa} receiver is approximately 3 Jy across the band.

%%% SENSITIVITY ENDS   %%%

%%% SKY COVERAGE BEGINS %%%

Since ALFABURST was installed, the majority of ALFA observation time is
allocated for the AGES \citep{2006MNRAS.371.1617A} and PALFA
\citep{2006ApJ...637..446C} surveys. The AGES survey pointing is primarily off
the galactic plane, thus there is little dispersion and scattering due to the galactic
\gls*{ism}. PALFA is a pulsar search survey with pointings near the galactic
plane. These lines of sight can introduce significant dispersion due to the
\gls*{ism}. We search out to a DM of 10000 which is well beyond the maximum
galactic dispersion, even when \gls*{igm} dispersion is accounted for from
sources of cosmological distances.

%TODO: figure of sky coverage

%%% SKY COVERAGE ENDS   %%%

% Introduction/Verification of system
%   * detection of bright single pulse events from known pulsars
%   * statement: no FRBs detected

\section{Event Prioritization and Classification}

% Event Classification and Prioritization
%   * priority classifier model
%       * labelled data
%       * wide feature selection + classifier
%       * deep feature selection + classifier
%   * event statistics

\section{Implied Low-Flux FRB Rates}

% Low flux FRB limits from ALFABURST
%   * Do we expect to see the known frbs at a further distance which we are probing?
%   * limits based on non-detection, observation time, and FRB standard candle model
%   * possible reasons: star formation rate, scintillation, steep spectrum
%   * implied FRB Luminosity function: cosmological star formation rate, peaks
%   around z=1? write notebook to produce rates
%   * implications in relation to current single-event FRBs and the repeater: limited frequency band, scintillation
%   * relation to ASKAP detections
%   * relations to Law et al multi-frequency observations: band-limited source,
%   but no astrophysical source
%   * relation to Macquart and Johnston paper: rate off/on the plane is
%   different due to scintillation regimes
%   * repeater and low freq frb searches: spectra are not pulsar like. may be
%   due to strong scintillation

\bibliographystyle{mnras}
\bibliography{alfaburst.bib} 

\bsp	% typesetting comment
\label{lastpage}
\end{document}

% End of mnras_template.tex
